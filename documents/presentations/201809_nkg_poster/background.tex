In geodesy we often deal with lists of coordinates and/or velocities of ground
markers and GNSS stations. These lists come in many shapes and sizes. Often the
points are represented in text files in a structured format but they can just as
easily be presented as a binary file in various formats. The files can be
generated by widely used software such as Bernese or perhaps by software only
meant for internal use in a national geodetic office.

An example of the latter is the KMS-format used in Denmark. KMS files come as
unstructured text files that are particularly difficult to parse by standard
text import tools. Similarly much work in Norway is based on the SOSI and FRI
formats. Converting between various file formats is often challenging making
data exchange from one software to another difficult.

One possible solution would be a standard file format for exchange of geodetic
data. Currently no definitive standard file format for exchange of coordinates
and velocities has been agreed upon in the geodetic community. While such a
format would indeed be nice to have, it is near impossible to get to a stage
where a single format is agreed as the work to implement it everywhere would be
immense.

Instead, we propose a new software that seamlessly converts between all common
formats. The inspiration to this approach comes from tools such as GDAL and
PDAL used in the world of remote sensing with great success. The new software
for converting between geodetic file formats has been named \textbf{Posetta},
obviously a play on words, in that this is a sort of Rosetta Stone for
positions.

%% Posetta is a NKG collaboration that so far only exists as a prototype. At the
%% time of writing Posetta can only convert between a few different file formats
%% but the architectural foundation has been created and adding new file formats is
%% fairly easy. We hope this tool will become useful for all geodesist within the
%% NKG and eliminate the need for reformatting files manually in the future.

\endinput
