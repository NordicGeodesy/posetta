Here are some examples of how to use Posetta as a command line program:

\textbf{Get help and information about the program}

\begin{verbatim}
  $ posetta -h
\end{verbatim}

\textbf{Convert file from XYZ format to CSV format}

Options \texttt{-f}, \texttt{-F} are used to specify which file and format to convert from. Options \texttt{-t}, \texttt{-T} specify which file to convert to.

\begin{verbatim}
  $ posetta -f example.xyz -F xyz -t example.csv -T csv
\end{verbatim}

\textbf{Combine with Proj to transform coordinates in a SOSI file}

If no input or output file is specified, Posetta will read from \texttt{stdin} or write to \texttt{stdout} respectively. Posetta can therefore be combined with other programs in a pipeline. The following uses Proj (\texttt{cct}) to transform from UTM32 coordinates to longitude and latitude.

\begin{verbatim}
  $ posetta -f utm32.sosi -F sosi -T proj \
      | cct +proj=utm +zone=32 +ellps=GRS80 \
      | posetta -F proj -t longlat.sosi -T sosi
\end{verbatim}

\endinput
